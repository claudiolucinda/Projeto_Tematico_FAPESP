%% abtex2-modelo-projeto-pesquisa.tex, v-1.7.1 laurocesar
%% Copyright 2012-2013 by abnTeX2 group at http://abntex2.googlecode.com/ 
%%
%% This work may be distributed and/or modified under the
%% conditions of the LaTeX Project Public License, either version 1.3
%% of this license or (at your option) any later version.
%% The latest version of this license is in
%%   http://www.latex-project.org/lppl.txt
%% and version 1.3 or later is part of all distributions of LaTeX
%% version 2005/12/01 or later.
%%
%% This work has the LPPL maintenance status `maintained'.
%% 
%% The Current Maintainer of this work is the abnTeX2 team, led
%% by Lauro César Araujo. Further information are available on 
%% http://abntex2.googlecode.com/
%%
%% This work consists of the files abntex2-modelo-projeto-pesquisa.tex
%% and abntex2-modelo-references.bib
%%

% ------------------------------------------------------------------------
% ------------------------------------------------------------------------
% abnTeX2: Modelo de Projeto de pesquisa em conformidade com 
% ABNT NBR 15287:2011 Informação e documentação - Projeto de pesquisa -
% Apresentação 
% ------------------------------------------------------------------------ 
% ------------------------------------------------------------------------

\documentclass[
	% -- opções da classe memoir --
	12pt,				% tamanho da fonte
	openright,			% capítulos começam em pág ímpar (insere página vazia caso preciso)
	twoside,			% para impressão em verso e anverso. Oposto a oneside
	a4paper,			% tamanho do papel. 
	% -- opções da classe abntex2 --
	%chapter=TITLE,		% títulos de capítulos convertidos em letras maiúsculas
	%section=TITLE,		% títulos de seções convertidos em letras maiúsculas
	%subsection=TITLE,	% títulos de subseções convertidos em letras maiúsculas
	%subsubsection=TITLE,% títulos de subsubseções convertidos em letras maiúsculas
	% -- opções do pacote babel --
	english,			% idioma adicional para hifenização
	french,				% idioma adicional para hifenização
	spanish,			% idioma adicional para hifenização
	brazil,				% o último idioma é o principal do documento
	]{abntex2}

% ---
% PACOTES
% ---

% ---
% Pacotes fundamentais 
% ---
\usepackage{cmap}				% Mapear caracteres especiais no PDF
\usepackage{lmodern}			% Usa a fonte Latin Modern
\usepackage[T1]{fontenc}		% Selecao de codigos de fonte.
\usepackage[utf8]{inputenc}		% Codificacao do documento (conversão automática dos acentos)
\usepackage{indentfirst}		% Indenta o primeiro parágrafo de cada seção.
\usepackage{color}				% Controle das cores
\usepackage{graphicx}			% Inclusão de gráficos
% ---

% ---
% Pacotes adicionais, usados apenas no âmbito do Modelo Canônico do abnteX2
% ---
\usepackage{lipsum}				% para geração de dummy text
% ---

% ---
% Pacotes de citações
% ---
\usepackage[brazilian,hyperpageref]{backref}	 % Paginas com as citações na bibl
\usepackage[alf]{abntex2cite}	% Citações padrão ABNT

% --- 
% CONFIGURAÇÕES DE PACOTES
% --- 

% ---
% Configurações do pacote backref
% Usado sem a opção hyperpageref de backref
\renewcommand{\backrefpagesname}{Citado na(s) página(s):~}
% Texto padrão antes do número das páginas
\renewcommand{\backref}{}
% Define os textos da citação
\renewcommand*{\backrefalt}[4]{
	\ifcase #1 %
		Nenhuma citação no texto.%
	\or
		Citado na página #2.%
	\else
		Citado #1 vezes nas páginas #2.%
	\fi}%
% ---

% ---
% Informações de dados para CAPA e FOLHA DE ROSTO
% ---
\titulo{Economia de Baixo Carbono em Mercados Emergentes: Uma agenda de pesquisa}
\autor{Claudio Ribeiro de Lucinda}
\local{Brasil}
\data{\today}
\instituicao{%
  Universidade de São Paulo -- USP
  \par
  Faculdade de Economia, Administração e Contabilidade -- FEA
  \par
  Departamento de Economia}
\tipotrabalho{Tese (Doutorado)}
% O preambulo deve conter o tipo do trabalho, o objetivo, 
% o nome da instituição e a área de concentração 
\preambulo{Projeto de Pesquisa -- Projeto Temático.}
% ---

% ---
% Configurações de aparência do PDF final

% alterando o aspecto da cor azul
\definecolor{blue}{RGB}{41,5,195}

% informações do PDF
\makeatletter
\hypersetup{
     	%pagebackref=true,
		pdftitle={\@title}, 
		pdfauthor={\@author},
    	pdfsubject={\imprimirpreambulo},
	    pdfcreator={LaTeX with abnTeX2},
		pdfkeywords={abnt}{latex}{abntex}{abntex2}{projeto de pesquisa}, 
		colorlinks=true,       		% false: boxed links; true: colored links
    	linkcolor=blue,          	% color of internal links
    	citecolor=blue,        		% color of links to bibliography
    	filecolor=magenta,      		% color of file links
		urlcolor=blue,
		bookmarksdepth=4
}
\makeatother
% --- 

% --- 
% Espaçamentos entre linhas e parágrafos 
% --- 

% O tamanho do parágrafo é dado por:
\setlength{\parindent}{1.3cm}

% Controle do espaçamento entre um parágrafo e outro:
\setlength{\parskip}{0.2cm}  % tente também \onelineskip

% ---
% compila o indice
% ---
\makeindex
% ---

% ----
% Início do documento
% ----
\begin{document}

% Retira espaço extra obsoleto entre as frases.
\frenchspacing 

% ----------------------------------------------------------
% ELEMENTOS PRÉ-TEXTUAIS
% ----------------------------------------------------------
% \pretextual

% ---
% Capa
% ---
\imprimircapa
% ---

% ---
% Folha de rosto
% ---
\imprimirfolhaderosto
% ---

% ---
% NOTA DA ABNT NBR 15287:2011, p. 4:
%  ``Se exigido pela entidade, apresentar os dados curriculares do autor em
%     folha ou página distinta após a folha de rosto.''
% ---

% resumo em português
\begin{resumo}
 Segundo a \citeonline[3.1-3.2]{NBR6028:2003}, o resumo deve ressaltar o
 objetivo, o método, os resultados e as conclusões do documento. A ordem e a extensão
 destes itens dependem do tipo de resumo (informativo ou indicativo) e do
 tratamento que cada item recebe no documento original. O resumo deve ser
 precedido da referência do documento, com exceção do resumo inserido no
 próprio documento. (\ldots) As palavras-chave devem figurar logo abaixo do
 resumo, antecedidas da expressão Palavras-chave:, separadas entre si por
 ponto e finalizadas também por ponto.

 \vspace{\onelineskip}
    
 \noindent
 \textbf{Palavras-chaves}: latex. abntex. editoração de texto.
\end{resumo}

% resumo em inglês
\begin{resumo}[Abstract]
 \begin{otherlanguage*}{english}
   This is the english abstract.

   \vspace{\onelineskip}
 
   \noindent 
   \textbf{Key-words}: latex. abntex. text editoration.
 \end{otherlanguage*}
\end{resumo}


% ---
% inserir o sumario
% ---
\pdfbookmark[0]{\contentsname}{toc}
\tableofcontents*
\cleardoublepage
% ---


% ----------------------------------------------------------
% ELEMENTOS TEXTUAIS
% ----------------------------------------------------------
\textual

% ----------------------------------------------------------
% Introdução
% ----------------------------------------------------------
\chapter*[Introdução]{Introdução}
\addcontentsline{toc}{chapter}{Introdução}


A UNFCCC (Convenção-Quadro das Nações Unidas sobre a Mudança do Clima) definiu, em seu Acordo de Paris, um objetivo ambicioso em limitar o aquecimento global em menos de dois graus centígrados, ao mesmo tempo em que busca esforços para limitar o aumento a 1,5 graus. Para isso, seria necessário reduzir as nossas emissões de CO2 em aproximadamente 45\% (em comparação aos níveis de 2010) até 2030\footnote{\url{https://unfccc.int/process-and-meetings/the-paris-agreement/katowice-climate-package}}. Mesmo um objetivo menos ambicioso, de limitar o aumento a dois graus centígrados implica em uma redução importante nas emissões de CO2. 

Conseguir objetivos tão ambiciosos é especialmente díficil para países emergentes, que se encontram no meio da transição para níveis de renda similares aos de economias mais desenvolvidas. Além disso, em muitos casos, países emergentes possuem limitações institucionais que tornam a implementação de alternativas já utilizadas em países mais desenvolvidas mais complexa. Neste sentido, o objetivo do presente projeto é discutir e estudar os desafios associados com a transição para uma Economia de Baixo Carbono em uma Economia emergente como o Brasil. 

Evidentemente, as limitações de um Projeto Temático da FAPESP impõem limites ao que pode ser estudado e analisado no contexto desta pergunta. Para isso, portanto, iremos nos focar em dois eixos principais:

\begin{itemize}
	\item Políticas de Incentivo ao Uso Eficiente da Infraestrutura Urbana
	\item Políticas de Incentivo ao Consumo de Produtos Duráveis Eficientes no Consumo de Energia
\end{itemize}

Estes dois eixos principais foram escolhidos pela sua relevância e por serem temas que perpassam as agendas de pesquisa de todos os pesquisadores principais da equipe. 

A relevância do primeiro dos temas é ressaltada pelo fato que a urbanização (entendida como o aumento da participação da população urbana no total) é fortemente correlacionada com o crescimento econômico. Não há evidência de países que alcançaram elevadas rendas per capita ou elevaram suas taxas de crescimento sem uma urbanização substancial e frequentemente muito rápida. Existe uma relação robusta entre urbanização e renda per capita. Neste sentido, é de se esperar que países que ainda não completaram sua transição para o desenvolvimento devam experimentar um forte movimento de urbanização. Tal urbanização acaba por gerar problemas, pois nem sempre a infraestrutura urbana consegue se adaptar adequadamente à utilização adicional neste processo. Em especial, a questão do congestionamento seria enfrentada aqui.


A relevância do segundo aspecto reside no fato que o crescimento econômico acaba por levar a um aumento na disseminação de bens duráveis na economia. Dentre estes bens duráveis, um aspecto importante é a troca (\textit{tradeoff}) entre a decisão de compra do bem e a sua utilização. No que diz respeito à utilização do bem, temos que para bens duráveis um dos maiores custos de utilização é a energia. 

Finalmente, este projeto temático também tem por objetivo incentivar a articulação de um grupo de pesquisadores em várias instituições, tanto no Brasil quanto no Exterior. Em especial, contamos com a participação dos seguintes pesquisadores:

\begin{itemize}
 	\item Prof. Dr. Rodrigo Menon Simões Moita (Insper/Brasil)
 	\item Prof. Dr. Cristian Huse (University of Oldemburg/Alemanha)
 	\item Prof. Dr. Antonio Bento (University of Southern California/EUA)
 	\item Prof. Dr. Ariaster Baumgratz Chimeli (USP/Brasil)
 \end{itemize} 

Todos eles já são pesquisadores estabelecidos na área e com quem tenho uma parceria de pesquisa inclusive financiada por projetos da FAPESP. Em especial, os seguintes pedidos de financiamento da FAPESP foram essenciais para esta parceria:

% Inserir projetos já financiados pela FAPESP

Neste sentido, o Projeto Temático busca consolidar uma parceria que já vem acontecendo há alguns anos, permitindo que esta participação se torne mais regular. Além disso, o presente projeto  irá permitir a consolidação de um grupo de pesquisa CNPQ, chamado \textit{Grupo de Pesquisa em Organização Industrial}, com espaço e infraestrutura física.

% ----------------------------------------------------------
% Capitulo de textual  
% ----------------------------------------------------------
\chapter{Elementos textuais}

\lipsum[1-10]

% ----------------------------------------------------------
% Capitulo com exemplos de comandos inseridos de arquivo externo 
% ----------------------------------------------------------

%\include{abntex2-modelo-include-comandos}

\chapter{Desafios Científicos e os Métodos para Enfrentá-los}

Em um projeto de fôlego como este, os desafios científicos são de duas naturezas distintas. O primeiro tipo de desafios a serem enfrentados são os problemas de natureza mais técnica e específicos aos diferentes sub-projetos nas agendas de pesquisa. Estes desafios serão detalhados no próximo capítulo, onde os diferentes sub-projetos serão detalhados.

O segundo tipo de desafios envolve a construção de um conjunto de bases de dados relevantes para os diferentes projetos e organizá-los de forma a permitir a utilização eficiente dentro dos diferentes projetos. 

\chapter{Estrutura dos Sub-Projetos}


\section{Políticas de Incentivo ao uso eficiente da infraestrutura urbana}



\subsection*{Pedágio Urbano}

A velocidade média do trânsito no município de São Paulo nos horários de pico vem se reduzindo ao longo das últimas décadas, passando de 26 km/h em 1980 para 22 km/h em 1991, 20 km/h em 2000 e chegando a 16 km/h em 2008, de acordo com a Companhia de Engenharia de Tráfego (CET) de São Paulo . Em uma tentativa de melhorar essa situação, foi criado, em 1997, o Rodízio Municipal na cidade de São Paulo (ou “Operação Horário de Pico”), pela lei  nº 12.490, a fim de reduzir o número de veículos nos horários de maior circulação.

O  rodízio consiste na proibição de circulação de alguns veículos, de acordo com o dígito final da placa, na região chamada de Centro Expandido. Entre segunda e sexta-feira, nos horários de pico da manhã (entre 7h e 10h) e tarde (entre 17h e 20h), dois finais de placas, a cada dia útil, não podem circular na área estipulada. No entanto, de acordo com os dados acima, o rodízio acabou se tornando apenas uma medida paliativa e a situação do trânsito no município continuou piorando.

Como observam \cite{Camara2004}, o efeito do rodízio foi sendo mitigado devido, principalmente, ao aumento da frota de veículos, que passou de 3,5 milhões em 1997, para quase 5 milhões em 2003, atingindo mais de 7 milhões em dezembro de 2012 no município, de acordo com o Departamento Estadual de Trânsito de São Paulo (DETRAN-SP). Em relação aos automóveis, estes atingiram 5,3 milhões em 2012 (ou 72\% da frota), chegando a uma relação de quase 1 automóvel para 2 habitantes. O crescimento no número de automóveis tem sido de aproximadamente 3,5\% ao ano desde 2008. Alternativamente, as motocicletas têm apresentado crescimento médio de 8\% ao ano no mesmo período, enquanto a frota de ônibus municipais permanece em torno de 15.000 desde 2006, segundo a São Paulo Transportes (SP-TRANS), autarquia responsável pelo planejamento e gerenciamento do tranposte público por ônibus no município.

Dado esse quadro, surge atualmente um debate a respeito da imposição de um pedágio urbano no município, sobre a mesma área em que já vigora o rodízio. O Projeto de Lei 01-0316/2010 do Vereador Carlos Apolinário, versando sobre o tema, foi considerado legal em abril de 2012 pela Comissão de Constituição, Justiça e Legislação Participativa (CCJ) da Câmara Municipal de São Paulo, porém não tramitou nas demais comissões da Câmara, provavelmente devido a 2012 ter sido um ano de eleições e pelo receio de uma possível impopularidade da medida.

A impopularidade e a dificuldade política do assunto é fato também em outros países do mundo. Na cidade de Estocolmo (Suíça), por exemplo, o pedágio urbano vinha sendo cogitado desde 1992, com o que ficou conhecido como Dennis Agreement, mas por questões políticas e insatisfação da população a ideia foi abandonada em 1997. Apenas em 2006, após um período de teste, o esquema foi definitivamente adotado no município \cite{Eliasson2008}. Além disso, outras cidades como Hong-Kong (China),  Nova Iorque (Estados Unidos) e Edimburgo (Reino Unido) tentaram aprovar projetos de pedágio urbano, mas não obtiveram sucesso.

Mesmo em São Paulo, \cite{Camara2004} lembram que a ideia de um pedágio urbano já tinha sido levada em consideração no âmbito da discussão pré rodízio, mas foi descartada devido à forte rejeição do público, captada por pesquisas realizadas em 1996, e críticas da imprensa. De fato, em 1995 foi proposta a implantação de uma cobrança sobre a circulação nas vias expressas do município, mas a medida foi considerada ilegal e não vingou.

\section{Políticas de Incentivo ao Consumo de Produtos Duráveis Eficientes no Consumo de Energia}




\subsection*{Flex Fuel em um mercado emergente: Bem-Estar e Efeitos de Política}

Uma das mais recentes inovações da indústria automobilística brasileira foi a introdução dos chamados carros flex-fuel, capazes de operar usando vários tipos de combustível. O primeiro modelo foi lançado em 2003 pela Volkwagen, como parte das comemorações dos seus 50 anos de atividade no Brasil, com um sistema de injeção eletrônica de combustível desenvolvido pela Bosch, que consiste em um sensor que detecta automaticamente a composição de combustível e ajusta adequadamente o motor.

Esta inovação é o resultado de um processo iniciado na última década do século XX,  como legado de um programa governamental dos anos 70: o Programa ProÁlcool. Este progresso claramente beneficiou os consumidores, ampliando suas possibilidades de escolha, ao permitir ajustar precisamente tanto o desempenho desejado quanto as despesas com combustível. Uma vez que o consumo de combustível em um carro difere  caso ele use gasolina ou álcool, prefere-se a gasolina como combustível quando o preço do etanol é  superior a 70\% do preço da gasolina. Se o preço do álcool estiver abaixo deste limite, a escolha deste combustível é mais vantajosa.

Ainda que o carro multicombustível seja uma inovação brasileira, rapidamente automóveis com estas características começaram a se disseminar pelo mundo. Um lugar em que esta tecnologia rapidamente se disseminou foi a Suécia, em que atualmente uma parcela importante dos novos automóveis já possui a tecnologia. Naquele país, além dos incentivos fiscais, há a interação do incentivo fiscal com a eliminação da congestion tax, imposto desenhado para desestimular tráfego nas grandes cidades suecas, como já discutido em \cite{Huse2013} e em \cite{Eliasson2008}.

A relevância deste tema faz com que seja um atraente tema de pesquisa aplicada na área de Economia Industrial, com excelentes possibilidades de, em primeiro lugar, proporcionar avanços significativos no conhecimento na área de estimação de demanda. Além disso, em uma linha de pesquisa mais aplicada, há um potencial de publicação grande com os resultados da estimativa de benefício ao consumidor para a introdução do carro flex-fuel, bem como a comparação destes resultados brasileiros com os análogos da Suécia, como de \cite{Huse2013}.

% ---
% Finaliza a parte no bookmark do PDF, para que se inicie o bookmark na raiz
% ---
\bookmarksetup{startatroot}% 
% ---


\chapter{Justificativa para os Itens de Despesa}

\section{Serviços de Terceiros}

\begin{itemize}
\item Serviço de Revisão e Edição de Textos em Inglês
\item Serviço de Base para a Realização de Seminários Internacionais
\item Serviço de Empresa de TI


\end{itemize}

\section{Bolsas}

\begin{itemize}
\item 10 (dez) Bolsas de Iniciação Científica
\item 1 (uma) Bolsa de Pós-Doutoramento
\item 2 (duas) Bolsas de Doutorado Direto
\end{itemize}

\section{Material Permanente}

\subsection{Nacional}
\begin{itemize}
\item Servidor com 16GB e cinco usuários
\item Dois Notebooks com 6GB cada
\end{itemize}

\subsection{Exterior}
Licença Stata 13/MP

\section{Despesas de Transporte}
Ida CRL a Suécia 1 vez por ano

\lipsum[1-10]

% ---
% Conclusão
% ---
\chapter*[Considerações finais]{Considerações finais}
\addcontentsline{toc}{chapter}{Considerações finais}

\lipsum[31-33]

% ----------------------------------------------------------
% ELEMENTOS PÓS-TEXTUAIS
% ----------------------------------------------------------
\postextual

% ----------------------------------------------------------
% Referências bibliográficas
% ----------------------------------------------------------
\bibliography{abntex2-modelo-references}


\end{document}
